% arara: pdflatex

%%%%%%%%%%
%  DRC Abstract Template for LaTeX Users
%  Latest Revision: 11 Jan 2016
%  Author: Samuel James Bader
%
%  README:
%   - This template is provided as a convenience for LaTeX users, and may not
%     suit every user's needs.  Feel free to modify as you see fit, so long as
%     you stay within the rules explicitly given for abstracts.
%   - This template supports both single-column and double-column layouts.
%     You need only change one line to switch, see below under ``IF YOU WANT
%     A TWO-COLUMN LAYOUT.''
%   - This template should work on common LaTeX compilers, but if this fails
%     to compile on your system due to font errors, see below where ``Times
%     New Roman'' is included and try a different one of the given options.
%
%%%%%%%%%%

%%%%%%%%%%
% YOU CAN IGNORE EVERYTHING FROM HERE UNTIL THE \begin{document}
%%%%%%%%%%
\documentclass[10pt]{article}

% Margins must be 1in all around
\usepackage[margin=1in]{geometry}

% Hyperlinks are blue
\usepackage[colorlinks=true, urlcolor=blue]{hyperref}

% URLs don't need to be mono-spaced
\urlstyle{same}

% Short form \email{} for giving email links
\newcommand{\email}[1]{\href{mailto:#1}{\underline{#1}}}

% Get math utilities
\usepackage{amsmath}
\usepackage{amsthm}
\usepackage{amssymb}

% Encodings
\usepackage[T1]{fontenc}
\usepackage[utf8]{inputenc}

% DRC abstracts use the Times New Roman font, and there are
% multiple ways to include this in LaTeX.  Option 1 should work
% for most LaTeX installations; however depending on your set-up,
% you may prefer Option 2 or 3. Uncomment just one option:

% Times New Roman: OPTION 1 (generally recommended)
\usepackage{newtxtext,newtxmath}
\DeclareTextCommandDefault{\textbullet}{\ensuremath{\bullet}}
% Times New Roman: OPTION 2 (fallback for older installations)
%\usepackage{mathptmx}
% Times New Roman: OPTION 3 (for XeLaTeX or LuaLaTeX users)
%\usepackage{fontspec}\setmainfont{Times New Roman}

% Shrink LaTeX line spacing a little to mirror Word template
\usepackage{setspace}
\setstretch{.955}

% Customize the author/affiliation section
\usepackage{authblk}

% No space between author and affiliation lists
\setlength{\affilsep}{0em}

% Authors in font size 12
\renewcommand{\Authfont}{\large}

% Affiliations in font size 10
\renewcommand{\Affilfont}{\itshape\normalsize}

% Define a command for author contact information
% (Wedge it into a ``affiliation'' line with no number.)
\newcommand{\authcontact}[2]{
  \affil[ ]{\textit{Email: \email{#1} / Phone: #2 }}
}

% Customize \maketitle to get rid of extra spacings and enforce font commands
\makeatletter
    \def\@maketitle{%
  \newpage
  \begin{center}%
  \let \footnote \thanks
  {\fontsize{14pt}{16.1pt}\selectfont \textbf{\@title}\par }%
    {\vspace{-3pt}%
      \begin{tabular}[t]{c}%
        \@author
      \end{tabular}\par
    }%
  \end{center}%
  \vspace{-1.2\baselineskip}
}
\makeatother

% Section headers are just bold normal-sized text
\usepackage[tiny]{titlesec}
\titlespacing*{\section}{0em}{\baselineskip}{0em}

% Allow multiple columns
\usepackage{multicol}
\setlength{\columnsep}{.5em}
\newif\ifdoublecol\doublecolfalse

% No numbered page footer
\pagestyle{empty}

% Keep numbered/bulleted lists compact
\usepackage{enumitem}
\setlist{nosep}

% Figures
\usepackage{graphicx}

% Label with Fig rather than figure.
\renewcommand{\figurename}{Fig.}

% Reduce line spacing in the bibliography
\usepackage{etoolbox}
\apptocmd{\thebibliography}{\setlength{\itemsep}{0em}}{}{}

%%%%%%%%%%
% START PAYING ATTENTION NOW
%%%%%%%%%%
\begin{document}

% IF YOU WANT A TWO-COLUMN LAYOUT, uncomment the following line
%\doublecoltrue

% Your TITLE goes here:
\title{The Design and Analysis of HyperFETs}

% AUTHOR LIST, with numbers indicating affiliation
\author[1]{Samuel James Bader}
\author[2]{Debdeep Jena}

% AFFILIATION LIST, with numbers to match the author list
\affil[1]{Department of Applied Physics, Cornell University, Ithaca, NY}
\affil[2]{Departments of ECE and MSE, Cornell University, Ithaca, NY}

% AUTHOR CONTACT INFORMATION here: email and phone number.
\authcontact{sjb353@cornell.edu}{(607)-255-1450 [Asst.]}
\maketitle
\thispagestyle{empty}

% Makes a double column layout *if specified above*
\ifdoublecol\begin{multicols}{2}\fi

% YOUR TEXT HERE
\section*{Introduction}
HyperFETs are pretty cool.

\section*{What did I do}
Some math

\section*{Significance}
Why did I do it?

% END YOUR TEXT

% Closes the double-column layout *if specified above*
\ifdoublecol\end{multicols}\fi

% Skip to bottom of page
\vfill

% Start double-column for References, 9pt font
\setlength{\multicolsep}{0em}
\begin{multicols}{2}
{\fontsize{9pt}{9pt}\selectfont

% Prevent extra spaces following periods in bibliography
\frenchspacing

% Prevents the bibliography from being titled
\renewcommand{\section}[2]{}
\begin{thebibliography}{9} 

% YOUR REFERENCES HERE
\bibitem{Franklin_2012} 
Reference 1
\bibitem{Yan_2015}
Reference 2
\bibitem{Salas_2015}
Reference 3
\bibitem{Han_2011}
Reference 4
% END YOUR REFERENCES

% Done with bibliography
\end{thebibliography}
}
\end{multicols}

% Now Figures Page
\pagebreak

% A figure containing two graphics in minipages
%\begin{figure}[!ht]
%  \centering
%  \begin{minipage}[t]{.49\textwidth}
%    \includegraphics[width=\textwidth]{images/Fig1}
%    \caption{This is an example figure to show one method for creating the figures on this page.  Two minipages just smaller than half the pagewidth are placed in one ``figure'' environment.  Each minipage holds one graphic/caption, and the two are horizontally stacked.}
%    \label{fig:CNT}
%  \end{minipage}
%\hfill
%  \begin{minipage}[t]{.49\textwidth}
%    \includegraphics[width=\textwidth]{images/Fig2}
%    \caption{Of course, seeing as this \textit{is} LaTeX, there are many different methods you could use for placing/aligning figures if you want to get more precise.  For starters, check out  \url{http://tex.stackexchange.com/a/148445/39047} or \url{https://en.wikibooks.org/wiki/LaTeX/Floats,_Figures_and_Captions}.}
%    \label{fig:channel}
%  \end{minipage}
%\end{figure}

% A plain old figure
%\begin{figure}[!ht]
%  \centering
%    \includegraphics[width=\textwidth]{images/Fig3}
%    \caption{Of course, you don't have to get fancy.  Here's a basic figure just spanning the width of the page.  But do notice the individually labelled subparts, and the fact that there is no more text on this page beyond captions.}
%    \label{fig:length}
%\end{figure}


% Thank you for your abstract!
\end{document}
